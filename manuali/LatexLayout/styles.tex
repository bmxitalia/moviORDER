\hypersetup{
    colorlinks=true,
    citecolor=black,
    filecolor=black,
    linkcolor=black, % colore link interni
    urlcolor=blue  % colore link esterni
}

\fancypagestyle{plain}{
    \fancyhdr{}

    \lhead{\GroupName \\ \GroupEmail}
    \rhead{\thechapter \\ \DocTitle \hspace{0.5mm} v\DocVersion}

    \lfoot{\includegraphics[scale=0.03,keepaspectratio=true]{../LatexLayout/moviORDER.png}}
    \cfoot{}
    \rfoot{\thepage{}}

    % Visualizza una linea orizzontale in cima e in fondo alla pagina
    \renewcommand{\headrulewidth}{0.3pt}
    \renewcommand{\footrulewidth}{0.3pt}
}
\setlength{\headheight}{20pt}
\pagestyle{plain}

% Titolo dell'indice
\addto\captionsenglish{
\renewcommand{\contentsname}%
  {Indice}%
}

\lstset{
    extendedchars=true,  % lets you use non-ASCII characters
    inputencoding=utf8,   % converte i caratteri utf8 in latin1, richiede
    %\usepackage{listingsutf8} anzichè listings
    basicstyle=\ttfamily,% the size of the fonts that are used for the
    %code
    breakatwhitespace=false,     % sets if automatic breaks should only happen at
    %whitespace
    breaklines=true,     % sets automatic line breaking
    captionpos=t,% sets the caption-position to top
    commentstyle=\color{mygreen},   % comment style
    frame=none,       % adds a frame around the code
    keepspaces=true,    % keeps spaces in text, useful for keeping
    %indentation of code (possibly needs columns=flexible)
    keywordstyle=\bfseries,     % keyword style
    numbers=none,       % where to put the line-numbers; possible values
    %are (none, left, right)
    numbersep=5pt,      % how far the line-numbers are from the code
    numberstyle=\color{mygray}, % the style that is used for the line-numbers
    rulecolor=\color{black},    % if not set, the frame-color may be changed on
    %line-breaks within not-black text (e.g. comments (green here))
    showspaces=false,   % show spaces everywhere adding particular
    %underscores; it overrides 'showstringspaces'
    showstringspaces=false,     % underline spaces within strings only
    showtabs=false,     % show tabs within strings adding particular
    %underscores
    stepnumber=5,       % the step between two line-numbers. If it's 1,
    %each line will be numbered
    stringstyle=\color{red},    % string literal style
    tabsize=4,  % sets default tabsize
    firstnumber=1      % visualizza i numeri dalla prima linea
}

\titleformat*{\section}{\LARGE\bfseries}
\titleformat*{\subsection}{\Large\bfseries}
\titleformat*{\subsubsection}{\large\bfseries}